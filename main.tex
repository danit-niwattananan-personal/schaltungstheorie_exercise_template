\documentclass[a4paper,12pt]{article}

\usepackage{amsmath}
\usepackage{scrlayer-scrpage}
\usepackage{graphicx}
\usepackage{geometry}
\usepackage[export]{adjustbox}
\usepackage{array}
\makeatletter
\newcommand*{\rom}[1]{\expandafter\@slowromancap\romannumeral #1@}
\makeatother
\usepackage[european,straightvoltages]{circuitikz}
\usepackage[locale=DE]{siunitx}
\usepackage[ngerman]{babel}
\usepackage[T1]{fontenc}
\usepackage{tikz}
\usepackage{caption}
\usepackage{subcaption}
\usepackage{blindtext}
\usepackage{multicol}
\setlength{\columnsep}{1cm}
\def\mathbi#1{\textbf{\em #1}}
\usepackage{graphicx}

\geometry{
	a4paper,
	total={170mm,257mm},
	left=25mm,
	right = 25mm,
	bottom= 25mm,
	top = 40mm
}

\begin{document}
	\ihead{Schaltungstheorie Tutorübung WS 2022/23\\ 7.Tutorium am 14.12.2022}
	\ohead{Danit Niwattananan\\  dummyemail@tum.de}
	\setheadsepline{.5pt}
	\cfoot{\pagemark}
	\section*{Aufgabe 1: Resistiver Eintor}

    Gegeben sei folgende Schaltung:
    \begin{figure}[ht]
		\centering

        \begin{circuitikz}
            %allg. Eintor F
            \draw(-4, 3) to [ageneric = $\mathcal{F}$, v> = $u_\mathcal{F}$, i>= $i_\mathcal{F}$, o-o](-4, 0);
            
            \draw(-2,1.5) node{$\equiv$};
        
            %Spannungsquelle
            \draw(0,0) to [V<= $u_0$](0, 3);
    
            %R und Diode
            \draw(3, 3) to [R = $R_1$, o-](0, 3);
            \draw (0,0) to[D, l=$D_1$, -o] (3,0); 
            
            % Pfeile
            \draw(2.5,0) to [short, i>= $i_{\mathcal{F}}$](2,0);
            \draw(3,0) to [open, v<= $u_{\mathcal{F}}$](3,3);
        \end{circuitikz}
    \end{figure}

Die Diode arbeitet im Sperrbereich.

    \begin{itemize}
        \item[a)] Zeichnen Sie ....
        \item[b)] Berechnen Sie $u_{\mathcal{F}}$ in Abhängigkeit von ....
    \end{itemize}
    Nun arbeitet die Diode im Durchlassbereich.
    \begin{itemize}
        \item[c)] Geben Sie $u_0$, in Abhängigkeit von ... an.
        \item[d)] Warum ist es so?
    \end{itemize}
    Es gelten folgende Bauteilwerte: $R_1 = 5 k \Omega$

    \begin{figure}[ht]
    \centering
    \begin{tikzpicture}
    \draw[step=0.5cm,gray,very thin] (-3.4,-3.4) grid (3.4,3.4);
    \draw[thick,->] (-3.4,0) -- (3.7,0) node[anchor=north west] {$\frac{u_{\mathcal{F}}}{V}$};
    \draw[thick,->] (0,-3.4) -- (0,3.7) node[anchor=south east] {$\frac{i_{\mathcal{F}}}{A}$};
    \end{tikzpicture}
    \end{figure}

    \begin{itemize}
        \item[e)] Zeichnen Sie das $u_{\mathcal{F}}$-$i_{\mathcal{F}}$-Diagramm.
    \end{itemize}

    \newpage
    
    \section*{Aufgabe 2: Never Gonna Give You Up}
    Gegeben sei folgende Zweitor.
    \begin{figure}[ht]
        \centering
        \includegraphics{your_circuit.pdf}
    \end{figure}

Der Zweitor kann durch folgende Gleichung beschrieben werden.
    \begin{equation*}
    \begin{bmatrix}
    1A & 2A\\
    3A & 4A
    \end{bmatrix} \mathbf{u} +
    \begin{bmatrix}
    5V & 6V\\
    7V & 8V
    \end{bmatrix}  \mathbf{i} = \textbf{0}
    \end{equation*}
    \begin{itemize}
        \item[a)] Never gonna let you down
        \item[b)] Never gonna run around
        \item[c)] and desert you
        \item[d)] Welche Eigenschaften besitzen die Netzwerkelemente
    \end{itemize}

    \begin{center}
    \begin{tabular}{ | c | m{.5cm}| m{.51cm} | m{.5cm}| } 
      \hline
      & $\mathcal{F}_1$ & $\mathcal{F}_{2}$ & $\mathcal{F}_{3}$ \\
      \hline
      spannungsgesteuert & & &  \\ 
      \hline
      stromgesteuert & & &  \\ 
      \hline
      attribute3 & & &  \\ 
      \hline
      attribute4 & & &  \\ 
      \hline
    \end{tabular}
    \end{center}
    \textit{Du schaffst es! Viel Erfolg!}
\end{document}